% Span Equations
%
% File:         equations.tex
% Author:       Bob Walton (walton@acm.org)
% Date:		See \date below.

\documentclass[12pt]{article}

\usepackage{times}
\usepackage{amsmath}
\usepackage{makeidx}

\makeindex

\setlength{\oddsidemargin}{0in}
\setlength{\evensidemargin}{0in}
\setlength{\textwidth}{6.5in}
\raggedbottom

\setlength{\unitlength}{1in}

\pagestyle{headings}
\setlength{\parindent}{0.0in}
\setlength{\parskip}{1ex}

\newcommand{\CN}[2]%	Change Notice.
    {\hspace*{0in}\marginpar{\sloppy \raggedright \it \footnotesize
     $^{\mbox{#1}}$#2}}
    % Change notice.

\newcommand{\pagnote}[1]{$\,^{p\,\pageref{#1}}$}

\newcommand{\EOL}{\penalty \exhyphenpenalty}

\newenvironment{indpar}[1][0.3in]%
	{\begin{list}{}%
		     {\setlength{\itemsep}{0in}%
		      \setlength{\topsep}{0in}%
		      \setlength{\parsep}{1ex}%
		      \setlength{\labelwidth}{#1}%
		      \setlength{\leftmargin}{#1}%
		      \addtolength{\leftmargin}{\labelsep}}%
	 \item}%
	{\end{list}}


\begin{document}
        
\begin{center}
\Large
{\LARGE  \bf Boardwalk and Trail Bridge Span Calculations}
\\[2ex]
January 21, 2025
\\
{\tt walton@acm.org}
\end{center}

\bigskip

\section{Provenence}

This document contains a copy of the equations in

\begin{center}
{\em ASD/LRFD Structural Wood Design Solved Example Problems, 2005 Edition, \\
Problem 6, Span of a Floor Joist}
\end{center}

This is because this document is a bit expensive and not available on the web.

Southern Pine Reference Design Values are obtained from the Southern
Forest Products Association:

\begin{center}
{\tt www.southernpine.com/wp-content/uploads/2023/09/TABLE01\_L1.pdf}
\end{center}

Acton snow load is obtained from the MA government:

\begin{center}
{\tt https://www.mass.gov/doc/16table1604pdf/download}
\end{center}

\newpage

\section{Reference Design Values}

For \#1 Standard Southern Pine 2"-4" dimensional width.

$F_b$ = (bending, $C_D$ and $C_M[F_b]$ apply)
\hspace*{0.3in}\begin{tabular}[t]{rll}
stringer \\
dimensional \\
height & $F_b$ & $C_M[F_b]$ \\
\hline
2"-4" & 1,500 psi & 0.85 \\
5"-6" & 1,350 psi & 0.85 \\
8" & 1,250 psi & 0.85 \\
10" & 1,050 psi & 1.0 \\
12" & 1,000 psi & 1.0 \\
\end{tabular}

$F_v$ = 175 psi (shear parallel to grain, $C_D$ and $C_M[F_v]$ apply)

$E$ = 1,600,000 psi (modulus of elasticity, $C_M[E]$ applies)

$F_{c\perp}$ = 565 psi (compression perpendicular to grain,
    $C_M[F_{c\perp}]$ applies)

$C_M[F_b]$ (wet service factor)
    = 0.85 when $F_b >$ 1,150 psi, = 1.0 when $F_b \leq$ 1,150 psi

$C_M[F_v]$ (wet service factor) = 0.97

$C_M[E]$ (wet service factor) = 0.9

$C_M[F_{c\perp}]$ (wet service factor) = 0.67

$C_D$ (duration factor) = 1.6 (ten minutes); = 1.25 (seven days);
                        = 0.9 (permanent - dead load)

$C_t$, $C_L$, $C_F$, $C_{fu}$, $C_i$, $C_r$, $C_b$ = 1.0

deflection limit = span/240 (from MA construction code)

design bearing length = 1.5 in  (from boardwalk construction)

Acton snow load = 55 psf  (from MA government table)

\newpage
\section{Equations}

In the following:

\begin{indpar}
$w$ = stringer actual width in inches \\
$h$ = stringer actual height in inches \\
$span$ = stringer span in feet \\
$W$ = weight on stringer in pounds force per linear foot (lbf/ft)
\end{indpar}
When necessary $span$ in inches is computed using
\begin{indpar}
$span$ in inches  = $span$ in feet $\times$
                    $\dfrac{12~\mathrm{inches}}{1~\mathrm{foot}}$
\end{indpar}

\subsection{Momentum Capacity (Bending)}

$F_b' = C_D \cdot C_M[F_b] \cdot F_b$
\\[1.0ex]
$S_x = \dfrac{w\cdot h^2}{6}$
\\[1.0ex]
$M' = S_x \cdot F_b'$ (allowable moment)
\\[1.0ex]
$M_{load} = W \cdot \dfrac{span^2}{8}$
\\[1.0ex]
$M_{load} \leq M'$

\subsection{Shear}

$V_{load} = W \cdot \dfrac{span}{2}$
\\[1.0ex]
$F_v' = C_D \cdot C_M[F_v] \cdot F_v$
\\[1.0ex]
$f_v = 3 \cdot \dfrac{V_{load}}{2 \cdot w \cdot h}$
\\[1.0ex]
$f_v \leq F_v'$

\newpage

\subsection{Deflection}

$E' = C_M[E] \cdot E$
\\[1.0ex]
$W_{defl} = 1.5 \cdot W_{\mathrm{dead~load}} + W_{\mathrm{live~load}}$
\\[1.0ex]
$\Delta = \dfrac{5 \cdot W_{defl} \cdot span^4}
                {384 \cdot E' \cdot \dfrac{w \cdot h^3}{12}}$
		~~~~~(deflection)
\\[1.0ex]
$Delta \leq {\mathrm{deflection~limit}} = \dfrac{span}{240}$

\subsection{Bearing Length}

$P_{load} = W \cdot \dfrac{span}{2}$
\\[1.0ex]
$F'_{c\perp} = C_M[F_{c\perp}] \cdot F_{c\perp}$
\\[1.0ex]
$Area_{bearing} = \dfrac{P_{load}}{F'_{c\perp}}$
\\[1.0ex]
$Length_{bearing} = \dfrac{Area_{bearing}}{w}$
\\[1.0ex]
design bearing length $\geq Length_{bearing}$





\end{document}

