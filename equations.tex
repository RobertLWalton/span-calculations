% Span Equations
%
% File:         equations.tex
% Author:       Bob Walton (walton@acm.org)
% Date:		See \date below.

\documentclass[12pt]{article}

\usepackage{times}
\usepackage{makeidx}

\makeindex

\setlength{\oddsidemargin}{0in}
\setlength{\evensidemargin}{0in}
\setlength{\textwidth}{6.5in}
\raggedbottom

\setlength{\unitlength}{1in}

\pagestyle{headings}
\setlength{\parindent}{0.0in}
\setlength{\parskip}{1ex}

\newcommand{\CN}[2]%	Change Notice.
    {\hspace*{0in}\marginpar{\sloppy \raggedright \it \footnotesize
     $^{\mbox{#1}}$#2}}
    % Change notice.

\newcommand{\pagnote}[1]{$\,^{p\,\pageref{#1}}$}

\newcommand{\EOL}{\penalty \exhyphenpenalty}

\newenvironment{indpar}[1][0.3in]%
	{\begin{list}{}%
		     {\setlength{\itemsep}{0in}%
		      \setlength{\topsep}{0in}%
		      \setlength{\parsep}{1ex}%
		      \setlength{\labelwidth}{#1}%
		      \setlength{\leftmargin}{#1}%
		      \addtolength{\leftmargin}{\labelsep}}%
	 \item}%
	{\end{list}}


\begin{document}
        
\begin{center}
\Large
{\LARGE  \bf Boardwalk and Trail Bridge Span Calculations}
\\[2ex]
January 19, 2025
\end{center}

\bigskip

\section{Provenence}

This document contains a copy of the equations in

\begin{center}
{\em ASD/LRFD Structural Wood Design Solved Example Problems, 2005 Edition,
Problem 6, Span of a Floor Joist}
\end{center}

This is because this document is expensive.

Southern Pine Reference Design Values are obtained from the Southern
Forest Products Association:

\begin{center}
{\tt www.southernpine.com/wp-content/uploads/2023/09/TABLE01\_L1.pdf}
\end{center}

\newpage

\section{Reference Design Values}

For \#1 Standard Southern Pine 2"-4" wide.

$F_b$ = (bending, $C_D$ and $C_M[F_b]$ apply)
\hspace*{0.3in}\begin{tabular}[t]{rl}
stringer \\
dimensional \\
height & $F_b$\\
\hline
2"-4" & 1,500 psi \\
5"-6" & 1,350 psi \\
8" & 1,250 psi \\
10" & 1,050 psi \\
12" & 1,000 psi \\
\end{tabular}

$F_v$ = 175 psi (shear parallel to grain, $C_D$ and $C_M[F_v]$ apply)

$E$ = 1,600,000 psi (modulus of elasticity, $C_M[E]$ applies)

$F_\perp$ = 565 (compression perpendicular to grain, $C_M[F_\perp]$ and $C_b$
                 apply)

$C_M[F_b]$ (wet service factor)
    = 0.85 when $F_b >$ 1,150 psi, = 1.0 when $F_b \leq$ 1,150 psi

$C_M[F_v]$ (wet service factor) = 0.97

$C_M[E]$ (wet service factor) = 0.9

$C_M[F_\perp]$ (wet service factor) = 0.67

$C_D$ (duration factor) = 1.6 (ten minutes); = 1.25 (seven days);
                        = 0.9 (permanent - dead load)

$C_r$ = 1.0 (repetitive member factor)

$C_b$ = 1.0 (bearing area factor)


\end{document}

